\documentclass[10pt,paper=A4,paper=landscape,headinclude=true,DIV=20]{scrartcl}
%\usepackage[a4paper, landscape, left=2.5cm, right=2.5cm, top=2.5cm, bottom=2.5cm]{geometry}
\usepackage[ngerman]{babel}
%\usepackage{arev}
\usepackage[T1]{fontenc}
\usepackage[scaled]{uarial}
\renewcommand{\familydefault}{\sfdefault}
\usepackage{color}
\usepackage[utf8]{inputenc}
\usepackage{amsmath}
\usepackage{amsfonts}
\usepackage{amssymb}
\usepackage{siunitx}
\usepackage[headsepline]{scrlayer-scrpage}
\usepackage{graphicx}
\pagestyle{scrheadings}
\usepackage{xltabular}
\sisetup{locale = DE}
\usepackage{multirow}
\usepackage{hyperref}
%\usepackage{lscape}
% \usepackage{longtable}
\newcommand{\phyphox}{\textit{phyphox} }
\DeclareUnicodeCharacter{226}{\tests}
\newcommand{\zeit}[2]{#1 \newline \newline #2'}
\begin{document}
\begin{flushright}
Name: Heiko Schröter\\
\end{flushright}
\vspace{0em}
\begin{tabularx}{1\textwidth}{|X|X|X|}
\hline
\multicolumn{3}{|>{\bfseries\Large}X|}{ \vspace{0em}Unterrichtsentwurf}                                                                                                                                                                                                                                                                                                                                                                            \\                                                        
\hline
Richard-Hartmann-Schule & LF/Fach: & Thema: \\
Berufliches Schulzentrum &  & Einführung \\
für Technik III, Chemnitz & Angewandte Physik & Mechanische Wellen \\
\hline 
Klasse: & Planung einer Unterrichtseinheit:  & Datum:  \\ 
&&\\
FOS 12 & 90 min & 18. März 2021\\
\hline 
\end{tabularx}\label{tab}
\\\\ \vspace{1em}
\textbf{Ziel: Einführung in die mechanischen Wellen}
\begin{itemize}
\item Berechnung der Zeit die bleibt, um sich vor den gefährlichen Oberflächenwellen eines Erdbebens in Sicherheit zu bringen.
\item Wir nennen Gemeinsamkeit und Unterschied von Schwingung und Wellen.
\item Welche Wellenarten gibt es und wodurch sind sie gekennzeichnet?
\item Wie hängen Wellenlänge und Ausbreitungsgeschwindigkeit zusammen?
\item Wir ermitteln die Ausbreitungsgeschwindigkeit einer Schallwelle in Luft.
\item Wie kann eine Welle mathematisch beschrieben und dargestellt werden?
\item Berechnung der Zeit zwischen den schnellen Transversal- und Longitudinal-wellen und den Oberflächenwellen bei einem Erdbeben.
\end{itemize}
% eingefühte Tabelle
\begin{xltabular}{1\textwidth}{|p{2.4em}|p{14cm}|p{6cm}|X|}
\hline 
Zeit & Inhalt & Methodisch-didaktisches Vorgehen & {Notizen/ \newline Bemerkungen} \\
\hline\endhead
\zeit{9:00}{5}& \textbf{Stundeneröffnung} Begrüßung und Erläuterung des Stundenziels & LV $\rightarrow$ Beamer &\\
\hline
\zeit{9:05}{5}& \textbf{Einstieg} Vorführung der Simulation eines realen Erdbebens & LV $\rightarrow$ Beamer & \url{http://ds.iris.edu/seismon/swaves/index.php} \\
\hline
\zeit{9:10}{15}& \textbf{Stoffvermittlung} Gemeinsamkeit und Unterschied von Schwingung und Welle (Simulation eines gekoppelten Pendels, Erweiterung auf mehrere Schwinger, mathematische Beschreibung der fortschreitenden Welle und grafische Darstellung, physikalische Größen Wellenlänge $\lambda$ und Ausbreitungsgeschwindigkeit $c$ & LV $\rightarrow$ Beamer & \url{https://www.walter-fendt.de/html5/phde/coupledpendula_de.htm}, Demoexperiment Stabschwinger, \textit{gnuplot}\footnote{Gnuplot ist ein skript- bzw. kommandozeilengesteuertes Computerprogramm zur grafischen Darstellung von Messdaten und mathematischen Funktionen (Funktionenplotter).} \\
\hline
\zeit{9:25}{10}& \textbf{Stoffvermittlung} Wellenarten (Transversal-wellen, Longitudinal-wellen, Oberflächenwellen) & LV $\rightarrow$ Beamer & Demoexperiment Spiralfeder \\
\hline
\zeit{9:35}{5}& \textbf{Stoffvermittlung} Zusammenhang Ausbreitungsgeschwindigkeit und Wellenlänge \newline $c=\dfrac{\lambda}{T}\quad f=\dfrac{1}{T}$ & LV $\rightarrow$ Beamer & \\
\hline
\zeit{9:40}{20}& \textbf{Sicherung} Ausbreitungsgeschwindigkeit von mechanischen Wellen (Experiment Bestimmung der Schallgeschwindigkeit mit APP \textit{Phyphox}\footnote{Die App \phyphox wird von der RWTH Aachen entwickelt und steht allen Interessierten \textbf{kostenlos} zur Verfügung. \phyphox macht es möglich, mit den Sensoren des Smartphones zu experimentieren, Messwerte aufzunehmen und auszuwerten.}  & GA & Arbeitsblatt, Smartphone \\
\hline

\zeit{10:00}{10}& \textbf{Sicherung} Berechnung der Wellenlänge einer Schallwelle & EA, Tafel & Übungsaufgabe \\
\hline
\zeit{10:10}{10}& \textbf{Stoffvermittlung} Mathematische Beschreibung einer Welle $y(x,t)=\hat{y}\cdot\sin\left\lbrace \omega\cdot\left(t-\dfrac{x}{c}\right)\right\rbrace=\hat{y}\cdot\sin(\omega t-k t)$ (graphische Darstellung mittels \textit{gnuplot}) & LV $\rightarrow$ Beamer & nur bei Bedarf als Puffer\\
\hline
\zeit{10:20}{5}& \textbf{Sicherung} Beispielaufgabe: Berechnung der Zeit zwischen den schnellen Transversal-wellen und Longitudinal-wellen und den langsameren Oberflächenwellen beim Erdbeben & EA, Tafel & Beispielaufgabe evtl. als Hausaufgabe \\
\hline
\zeit{10:25}{5}& Fragen/ Wiederholung/ Feedback & &\\
\hline
\zeit{10:30}{0}& Pause & &\\
\hline
\end{xltabular}

\end{document}